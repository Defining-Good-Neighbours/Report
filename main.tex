\documentclass[a4paper,doc,11pt]{article}
%----------------------------------------------------------------------------------------
%	Paquetes y configuraciones
%----------------------------------------------------------------------------------------
\usepackage[numbers]{natbib}
\bibliographystyle{apalike}

\usepackage{amsfonts}
\usepackage{amsmath}
\usepackage{amssymb,amsthm}
\usepackage{enumerate}
\usepackage{enumitem}
\usepackage[utf8]{inputenc}
\usepackage[T1]{fontenc}
\usepackage{geometry}
\usepackage{hyperref}
\geometry{left=1.75cm,right=1.75cm,top=2.5cm,bottom=2.5cm}



\usepackage{url}
\def\UrlBreaks{\do\/\do-}
\usepackage{multirow}
\usepackage{multicol}
\usepackage{enumitem}
\usepackage{nicefrac}
\usepackage{graphicx}
\usepackage{stmaryrd}
\usepackage{dsfont}
\usepackage{bropd}
\usepackage{easybmat}
\usepackage{setspace}
\usepackage{comment}
\usepackage{mathpazo}
\usepackage{array}
\usepackage{commath}

\usepackage{sectsty}
\sectionfont{\centering\fontsize{13}{15}\selectfont}
\subsectionfont{\centering\fontsize{10}{10}\selectfont\scshape}

\newtheorem{theorem}{Theorem}[section]
\newtheorem{corollary}{Corollary}[theorem]
\newtheorem{proposition}{Proposition}[theorem]
\newtheorem{lemma}[theorem]{Lemma}
\newtheorem{definition}[theorem]{Definition}
\newtheorem{remark}[theorem]{Remark}
\newtheorem{example}[theorem]{Example}
\newtheorem{claim}{Claim}[subsection]




\usepackage[font=small]{caption}
\usepackage[font=small]{subcaption}
\captionsetup{subrefformat=parens}
\usepackage{booktabs} % nice headers for tables

\newcommand{\R}{\mathbb{R}}
\newcommand{\Z}{\mathbb{Z}}
\newcommand{\N}{\mathbb{N}}
\newcommand{\CC}{\mathcal{C}}
\newcommand{\llb}{\llbracket}
\newcommand{\rrb}{\rrbracket}

\DeclareMathOperator{\dom}{dom}
\DeclareMathOperator{\dist}{dist}
\DeclareMathOperator{\supp}{supp}
\setcounter{MaxMatrixCols}{20}


\SetLabelAlign{parright}{\parbox[t]{\labelwidth}{\raggedright#1}}
\allowdisplaybreaks


\usepackage[symbol]{footmisc}

\renewcommand{\thefootnote}{\fnsymbol{footnote}}


\linespread{1.38}

%---------------------------------------- Autoría ---------------------------------------- %
%\usepackage{titling}
%\predate{}
%\postdate{\vspace{-2\baselineskip}}



\title{\bf
    \Large
    Defining Good Neighbours Project
}
\author{
    Karolína Benková, Liz Howell, Andrés Miniguano Trujillo\footnote{ \texttt{\{K.Benkova, L.Howell, Andres.Miniguano-Trujillo\}@ed.ac.uk} }
}
\date{}


\begin{document}
%\pagenumbering{Roman} 
\maketitle




%\newpage
%\tableofcontents


%%%%%%%%%%%%%%%%%%%%%%%%%%%%%%%%%%%%%%%%%%%%%%%%%%%
%\newpage
\section{Zinc uptake by rice through phytosiderophore secretion}
%\pagenumbering{arabic}

%%%%%%%%%%%%%%%%%
%%%%%%%%%%%%%%%%%


In the following section we will focus on the following nonlinear system of PDE's:
\begin{subequations}
\label{eq:system-Zinc}
\begin{align}
    \left( \theta + \frac{b_X}{1 + \kappa_X b_X^2 Y_L} \right) \partial_t X_L - \frac{\kappa_X b_X X_L}{(1+\kappa_X b_X Y_L)^2} \partial_t Y_L &=
    \nabla \cdot ( D_X \nabla X_L - \nu X_L  ) - g_X
    \\
    \left( \theta + \frac{b_Y}{1 + \kappa_Y b_Y^2 X_L} \right) \partial_t Y_L - \frac{\kappa_Y b_Y Y_L}{(1+\kappa_Y b_Y X_L)^2} \partial_t X_L &=
    \nabla \cdot ( D_Y \nabla Y_L - \nu Y_L  ) - g_Y;
\end{align}
where $X(r,z)$ and $Y(r,z)$ are subject to the boundary conditions:
\begin{align}
    D_X \partial_r X_L - \nu X_L &= \alpha X_L & \text{at } r = a, z \in [L - \delta L_X, L],
    \\
    D_Y \partial_r Y_L - \nu Y_L &= -F_Y(t) &\text{at } r = a, z \in [L - \delta L_Y, L],
\end{align}
where $L = L_0 + G t$. For the rest of the boundary, we have zero-flux conditions; i.e., in the case \( r = x\), this is just
\begin{align}
    D_X \partial_r X_L - \nu X_L = 0
    \qquad\text{and}\qquad
    D_Y \partial_r Y_L - \nu Y_L = 0.
\end{align}
Moreover, $r \in [a,x]$, with $x = (\pi L_V)^{1/2}$, and $z \in [0, L_{t_{\mathrm{max}} }]$.
\end{subequations}


%%%%%%%%%%%%%%%%%
%%%%%%%%%%%%%%%%%
\subsection{Variational formulation}

To find the variational formulation of system \eqref{eq:system-Zinc}, we will use Galerkin's method. Let's begin supposing that there is a $\mathcal{C}^2 (\bar \Omega)$ solution pair $(X_L,Y_L)$ and multiply the first equation by a $v \in \mathcal{C}^1 (\bar\Omega)$ function and introduce $A_X(Y_L) = \theta + \frac{b_X}{1 + \kappa_X b_X Y_L}$ and $B_X(X_L,Y_L) = \frac{\kappa_X b_X^2 X_L}{(1+\kappa_X b_X Y_L)^2}$. We then get
\begin{align}
    A_X(Y_L) \partial_t X_L v - B_X(X_L,Y_L) \partial_t Y_L v = \nabla \cdot (D_X \nabla X_L - \nu X_L) v - g_X v.
\end{align}
Integrating this, and using the Divergence Theorem, we have
\begin{align}
    \int\limits_\Omega
    A_X(Y_L) \partial_t X_L v &- B_X(X_L,Y_L) \partial_t Y_L v \dif \mu = 
    \int\limits_\Omega
    \nabla \cdot (D_X \nabla X_L - \nu X_L) v  \dif \mu
    -\int\limits_\Omega g_X v \dif \mu
    \\
    &=
    \int\limits_\Gamma
    (D_X \nabla X_L - \nu X_L) \cdot \vec{n} v
    \dif \sigma
    -\int\limits_\Omega
    (D_X \nabla X_L - \nu X_L) \cdot \nabla v  \dif \mu
    -\int\limits_\Omega g_X v \dif \mu;
\end{align}
where $\Gamma$ is the boundary of $\Omega$, $\vec{n}$ its normal vector and $\sigma$ the surface measure on $\Gamma$. 

Notice that we only need to analyse three segments of $\Gamma$: $\Gamma_{a_X} = \{a\}\times [L-\delta L_X,L]$, $\Gamma_{a_Y} = \{a\} \times [L - \delta L_Y, L]$, and $ \Gamma_0 = \Gamma \setminus (\Gamma_{a_X} \cup \Gamma_{a_Y})$. At $r = a$, we have that $ \vec{n} = (-1,0)$, thus
\begin{align}
    \int\limits_{\Gamma_{a_X}}
    (D_X \nabla X_L - \nu X_L) \cdot \vec{n} v    \dif \sigma
    =
    -
    \int\limits_{\Gamma_{a_X}}
    (D_X \partial_r X_L - \nu X_L) v    \dif \sigma
    =
    -\int\limits_{\Gamma_{a_X}}
    \alpha X_L v    \dif \sigma.
\end{align}
Similarly, in the equation for \(Y_L\) we have
\begin{align}
    \int\limits_{\Gamma_{a_Y}}
    (D_Y \nabla Y_L - \nu Y_L) \cdot \vec{n} v    \dif \sigma
    =
    \int\limits_{\Gamma_{a_Y}}
    F_Y(t) v    \dif \sigma.
\end{align}
For the other segments of the border, the surface area is zero. To see this, consider $r = x$, where we have that $ \vec{n} = (1,0)$, thus
\begin{align}
    \int\limits_{\Gamma_0 \cap \{r=x\}}
    (D_X \nabla X_L - \nu X_L) \cdot \vec{n} v    \dif \sigma
    =
    \int\limits_{\Gamma_0 \cap \{r=x\}}
    (D_X \partial_r X_L - \nu X_L) v    \dif \sigma
    =
    0.
\end{align}

As a result, we have that the following equation must be satisfied for every $v \in \mathcal{C}^1 (\bar\Omega)$:
\begin{align}
    \int\limits_\Omega
    A_X(Y_L) \partial_t X_L v - B_X(X_L,Y_L) \partial_t Y_L v \dif \mu
    &=
    -\int\limits_{\Gamma_{a_X}}    \alpha X_L v    \dif \sigma
    -\int\limits_\Omega        (D_X \nabla X_L - \nu X_L) \cdot \nabla v  \dif \mu
    -\int\limits_\Omega        g_X v \dif \mu.
\end{align}
Notice that by density, the weak form of the previous equation is the same over $H^1(\Omega)$.
Similarly, the variational form for the governing equation of $Y_L$ is
\begin{align}
    \int\limits_\Omega
    A_Y(X_L) \partial_t Y_L v - B_Y(Y_L,X_L) \partial_t X_L v \dif \mu
    &=
    \int\limits_{\Gamma_{a_Y}}     F_Y(t) v    \dif \sigma
    -\int\limits_\Omega        (D_Y \nabla Y_L - \nu Y_L) \cdot \nabla v  \dif \mu
    -\int\limits_\Omega        g_Y v \dif \mu 
\end{align}
for all \(v \in H(\Omega)\).



%%%%%%%%%%%%%%%%%
%%%%%%%%%%%%%%%%%
\subsection{Time discretisation}

To approximate the time derivative we will use a simple backward difference. First, we will discretise the time domain; then, let the superscript $n$ denote a quantity at time $t_n$, where $n$ is an integer counting time levels; e.g. $X_L^n$ is $X_L$ evaluated at time level $n$. This yields
\begin{equation}
    \left(\frac{\partial X_L}{\partial t}\right)^{n+1} \approx \frac{X_L^{n+1} - X_L^n}{\Delta t}.
\end{equation}
Inserting this in the variational formulation, we get the approximation
\begin{equation}
\begin{aligned}
    \int\limits_\Omega
    \big(
    A_X(Y_L^{n+1}) &(X_L^{n+1} - X_L^n) - B_X(X_L^{n+1},Y_L^{n+1}) (Y_L^{n+1} - Y_L^n)  
    \big) v \dif \mu
    \\
    &+
    \Delta t \int\limits_\Omega        (D_X \nabla X_L^{n+1} - \nu X_L^{n+1}) \cdot \nabla v  \dif \mu
    +
    \Delta t \int\limits_{\Gamma_{a_X^{n+1}}}    \alpha X_L^{n+1} v    \dif \sigma
    =
    -\Delta t \int\limits_\Omega        g_X v \dif \mu.
\end{aligned}
\end{equation}
In the case of $Y_L$, we proceed with the same time scale and get
\begin{equation}
\begin{aligned}
    \int\limits_\Omega
    \big(
        A_Y(X_L^{n+1}) & (Y_L^{n+1} - Y_L^n)  - B_Y(Y_L,X_L) (X_L^{n+1} - X_L^n)
    \big) v \dif \mu
    \\
    &
    + \Delta t\int\limits_\Omega        (D_Y \nabla Y_L^{n+1} - \nu Y_L^{n+1}) \cdot \nabla v  \dif \mu
    =
    \Delta t\int\limits_{\Gamma_{a_Y^{n+1}}}     F_Y(t_{n+1}) v    \dif \sigma
    -\Delta t\int\limits_\Omega        g_Y v \dif \mu.
\end{aligned}
\end{equation}

Notice that we further need to approximate the system at time \(t=0\). At this level, we will set \( (X_L^0,Y_L^0) \equiv (1\times 10^{-8},0)\) M.

%\[
%    \int\limits_\Omega u \pd{\varphi}{x_i} \dif x = -\int\limits_\Omega \pd{u}{x_i} \varphi \dif x,
%\]











%%%%%%%%%%%%%%%%%%%%%%%%%%%%%%%%%%%%%%%%%
\newpage

\section*{Availability of data, material, and code}
{
%\small

All the files and this document are available as in the following repository:
\begin{quote}
    \noindent \href{https://github.com/Defining-Good-Neighbours/}{\texttt{https://github.com/Defining-Good-Neighbours/}}
\end{quote}



}

%%%%%%%%%%%%%%%%%%%%%%%%%%%%%%%%%%%%%%%%%
%%%%%%%%%%%%%%%%%%%%%%%%%%%%%%%%%%%%%%%%%
%\newpage


\bibliography{Cites}

\end{document}



